\documentclass[14pt]{beamer}
\usetheme{Montpellier}
\usepackage[utf8]{inputenc}
\usepackage[english]{babel}
\usepackage{amsmath}
\usepackage{amsfonts}
\usepackage{amssymb}
\usepackage{graphicx}
\usepackage{tikz}

\author{Oumaima Al Qoh, Francisco Arrieta, Lucia Camenisch, Manuela Giansante, Emily Schmidt, Camille Beatrice Valera}
\title{Case Study B \\ Pain Relief Medication}
\date{April 28, 2023} 
%\subject{}
%\logo{}
%\institute{}

  
\setbeamertemplate{navigation symbols}{
\usebeamerfont{footline}
\insertframenumber / \inserttotalframenumber
}



\begin{document}


\begin{frame}
\titlepage
\end{frame}



\begin{frame}
\frametitle{Table of contents}
\tableofcontents
\end{frame}


\section{Objectives}
\subsection{First objective}
\begin{frame}
\frametitle{First objective}
\textit{Determine the minimum dosage amount of the drug that achieves \textbf{efficacy}.}

\bigskip

The dosage of a drug is said to be \textbf{efficient} if at least 50\% of the subjects are responding.
\end{frame}


\subsection{Second objective}
\begin{frame}
\frametitle{Second objective}
\textit{Detect whether a \textbf{synergy} exists between the two drugs.}

\bigskip

If the combined efficacy of the two drug dosages is greater than the sum of the individual efficacy of each drug for its respective dosage, there is \textbf{synergy}.

\end{frame}




\end{document}